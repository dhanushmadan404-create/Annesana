 <header>
     <div class="header_container">

       <img class="logo" src="/assets/annesana.png" alt="not able to see" onclick="window.location.href='/index.html'" />
       
       <img class="logo location" src="/assets/image.png" alt="not available" onclick="window.location.href='/pages/map.html'">
       
       
       <div class="right-align right">
         <a href="#" id="login">Log in</a>
         <div class="search-bar">
           <input type="text" placeholder="Search..." />
           <button>🔍</button>
          </div>
        </div>
      </div>
    </header>



css:

header {
  height: 70px;

  padding: 15px 30px;
  background: #e5a244; /* spicy orange-yellow */
  box-shadow: 0 2px 10px rgba(255, 140, 0, 0.3);
  border-bottom: 3px solid #e85d04;
  flex-wrap: wrap;
  overflow-y: hidden;
  
}
.header_container{
    display: flex;
  justify-content: space-between;
  align-items: center;
  position: relative;
  bottom: 20px;

}
.location{
  position: relative;
  left: 80px;
  border-radius: 50%;
  box-shadow: 5px 10px 20px rgb(46, 46, 46);
}
.location:hover{
  border: #d9480f solid 10px;
}

/* Search Bar */
.search-bar ,.right-align{
  display: flex;
  align-items: center;
  
}
.right{
  gap: 20px;
}

.search-bar input {
  padding: 8px 10px;
  width: 250px;
  border: 2px solid #e85d04;
  border-radius: 4px 0 0 4px;
  outline: none;
  background: #fff;
  color: #3b1f00;
}
.right-align a{
  color: #3f3f3e;
}
.right-align a:hover{
  color: #090908;
}

.search-bar button {
  background: #e85d04;
  color: #fff;
  border: none;
  padding: 8px 12px;
  border-radius: 0 4px 4px 0;
  cursor: pointer;
  font-weight: bold;
  transition: 0.3s;
}

.search-bar button:hover {
  background: #d9480f;
}
.logo {
  width: 100px;
  height: 100px;
    background: transparent;
    mix-blend-mode: multiply;
  transition: 0.3s ease;
}




class Food(Base):
    __tablename__ = "foods"
    
    food_id = Column(Integer, primary_key=True, index=True, autoincrement=True)
    vendor_id = Column(Integer, ForeignKey("vendors.vendor_id"), nullable=False)
    food_name = Column(String(255), nullable=False)
    food_image_url = Column(String(255))
    category = Column(String(100))
    latitude=Column(Float,nullable=False)
    longitude=Column(Float,nullable=False)
bro listen carefully first registration shoold be like this picture in this pic vendor enter vada ,sambar etc this all thing might added in a list it's separated by like this model food class so everything might have be separated so when vendor update the food list or location everything in model food class we should focus on food list now vendor remove like sambar then add idly then i separated everything  transfered to backend first  


% tHIS TRANSFER DATA 
const formData = new FormData(form);
  const formObject = Object.fromEntries(formData.entries());

  
  try {
    const response = await fetch("http://127.0.0.1:8000/users", {
      method: "POST",
      headers: {
        "Content-Type": "application/json",
      },
      body: JSON.stringify(formObject),
    });

    const data = await response.json();

    if (!response.ok) {
      console.error("Backend error:", data);
      alert(data.detail || "Submission failed");
      return;
    }

    console.log("Success:", data);
    alert("User created successfully!");
    form.reset();

  } catch (error) {
    console.error("Network Error:", error);
    alert("Server not reachable");
  }